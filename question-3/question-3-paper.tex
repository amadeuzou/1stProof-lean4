\documentclass[11pt]{article}

\usepackage[a4paper,margin=1in]{geometry}
\usepackage{amsmath,amsthm,amssymb}
\usepackage{hyperref}

\hypersetup{
  colorlinks=true,
  linkcolor=blue,
  citecolor=blue,
  urlcolor=blue
}

\newtheorem{theorem}{Theorem}[section]
\newtheorem{proposition}[theorem]{Proposition}
\newtheorem{corollary}[theorem]{Corollary}
\theoremstyle{definition}
\newtheorem{definition}[theorem]{Definition}
\newtheorem{assumption}[theorem]{Assumption}
\theoremstyle{remark}
\newtheorem{remark}[theorem]{Remark}

\newcommand{\Snlam}{S_n(\lambda)}
\newcommand{\rate}{\mathsf{rate}}

\title{Detailed Balance from Exchange Relations:\\
Formal Verification of the Stationary Measure Construction\\
for Multi-Species ASEP}
\author{Amadeu Zou\\\texttt{amadeuzou@gmail.com}}
\date{February 12, 2026}

\begin{document}
\maketitle

\begin{abstract}
We present a Lean 4 formalization of a stationary-measure construction for a ring multi-species ASEP model on restricted sectors.
The core theorem proves a strict local-to-global pipeline:
exchange relations of Macdonald type imply local detailed balance, which implies global balance and stationarity.
We then complete the probability layer by proving nonnegativity, normalization, and nontriviality after uniformization to a discrete-time kernel.
A central design principle is \emph{interface decoupling}: special-function identities are treated as explicit assumptions, while Markov-chain conclusions are derived with no hidden analytic steps.
This yields a reusable theorem schema suitable for literature transfer:
any polynomial family satisfying the interface assumptions induces a verified stationary distribution of the form \(F^*/P^*\).
Code: \url{https://github.com/amadeuzou/1stProof-lean4}.
\end{abstract}

\section{Introduction}
The multi-species ASEP (m-ASEP) on finite rings is a standard model in integrable probability and interacting particle systems.
In many constructions, explicit stationary weights are provided through exchange identities connected to Macdonald-type structures.
However, the full probability-theoretic chain is often spread across separate arguments:
local identities, positivity side conditions, normalization, and Markov-kernel validity.

This paper provides a machine-checked consolidation of that chain.
Our formal development is not aimed at replacing the classical formulas.
Instead, it provides a precise proof interface that isolates where special-function input is used and certifies the resulting stochastic conclusions.

\paragraph{Contributions.}
\begin{enumerate}
\item A formal local-to-global theorem:
exchange relations imply detailed balance and stationarity.
\item A formal probability completion:
nonnegative rates, normalized stationary law, and nontrivial transitions.
\item A literature-facing interface theorem:
under explicit assumptions on \(F^*\) and \(P^*\), we obtain a certified discrete Markov chain with stationary law \(F^*/P^*\).
\item A complete Lean 4 artifact with successful project build and no unfinished placeholders.
\end{enumerate}

\section{Model on Restricted Sectors}
\subsection{State space}
Fix \(m,n \ge 2\), and let \(\lambda\) be a bounded word of length \(n\) over species set \(\{0,\dots,m-1\}\), satisfying the restricted-sector conditions used in the formal model.
The state space is
\[
\Snlam := \{\sigma\lambda : \sigma \in \mathfrak{S}_n\},
\]
implemented as a finite subtype (\texttt{SnLambdaState}).

\subsection{Local transitions}
For an adjacent index \(a=(i,i+1)\), define
\[
\rate^+_a = t x_i - x_{i+1}, \qquad
\rate^-_a = x_i - t x_{i+1}, \qquad
\mathsf{ratio}_a = \frac{\rate^+_a}{\rate^-_a}.
\]
The local edge kernel (\texttt{edgeKernel}) contributes only when \(w'\) is obtained from \(w\) by the adjacent swap at \(a\).
The full continuous-time rate (\texttt{transitionRate}) is the finite sum over all adjacent \(a\).

\section{Exchange Relations and Balance Equations}
\subsection{Exchange assumption}
\begin{definition}[Exchange relation]
For a weight function \(\pi\), the exchange relation at adjacent inversion \(a\) is
\[
\pi(s_a w) = \mathsf{ratio}_a \,\pi(w).
\]
\end{definition}

\subsection{Local detailed balance}
\begin{proposition}[Local detailed balance]
\label{prop:local-db}
Under denominator nonvanishing and the exchange relation,
\[
\pi(w)\,q(w,s_aw)=\pi(s_aw)\,q(s_aw,w)
\]
holds for each adjacent swap edge.
\end{proposition}

\begin{proof}[Proof sketch]
The Lean theorem \texttt{Question3.local\_detailed} proves a direct algebraic identity by rewriting both sides into the same rational expression in \((x,t)\), using the explicit edge-rate formulas and the exchange ratio.
\end{proof}

\subsection{Global balance}
\begin{theorem}[Global balance from local balance]
\label{thm:global-balance}
If Proposition~\ref{prop:local-db} holds for all adjacent edges, then
\[
\sum_{u\in\Snlam}\pi(u)\,q(u,v)
=
\pi(v)\sum_{z\in\Snlam}q(v,z),
\qquad \forall v\in\Snlam.
\]
\end{theorem}

\begin{proof}[Proof sketch]
Summing the local edge equalities over adjacent indices and states gives the global identity.
This is formalized in \texttt{Question3.stationary\_global\_balance\_explicit}.
\end{proof}

\section{Probability Completion}
\subsection{Rate nonnegativity interface}
\begin{definition}[Rate-pair nonnegativity]
\texttt{RatePairNonneg} requires \(\rate^+_a \ge 0\) and \(\rate^-_a \ge 0\) for all adjacent \(a\).
\end{definition}

\begin{proposition}
Under \texttt{RatePairNonneg}, all edge kernels and all total transition rates are nonnegative.
\end{proposition}

\begin{proof}[Formal anchors]
See the nonnegativity lemmas in \texttt{Question3}
(listed explicitly in Section~6).
\end{proof}

\subsection{Uniformization and discrete kernel}
Given a finite continuous-time rate \(q\), define a bound \(B>0\) and
\[
P(u,v)=\frac{q(u,v)}{B}+\mathbf{1}_{u=v}
\left(1-\frac{\sum_z q(u,z)}{B}\right).
\]
The formal development proves:
\begin{enumerate}
\item \(P(u,v)\ge 0\),
\item \(\sum_v P(u,v)=1\),
\item stationarity transfer from \(q\) to \(P\),
\item existence of a strictly positive off-diagonal transition.
\end{enumerate}

These are implemented by
four kernel lemmas in
\texttt{Macdonald.Bridge.FinalTheorem}
(listed explicitly in Section~6).

\section{Literature Interface and Main Theorems}
\subsection{Interface theorem}
The structure \texttt{FstarCandidateOnRestricted} packages:
restricted-sector data, exchange-side hypotheses, a candidate \(\pi\), and nontrivial local dynamics.
The literature-facing closure theorem is:

\begin{theorem}[Conditional closure theorem]
\label{thm:literature-closure}
The literature-closure theorem states that:
if a candidate satisfies exchange assumptions, rate-pair nonnegativity, and
normalized nonnegative \(\pi\), then there exists a discrete-time Markov chain
on \(\Snlam\) whose stationary law is exactly \(F^*/P^*\), with nontrivial
off-diagonal dynamics.
\end{theorem}

\subsection{Repository-final bundled statements}
For the canonical restricted \(q=1\) model, the project provides explicit final theorems:
\begin{enumerate}
\item \texttt{question3\_complete\_restricted\_qOne} (continuous time),
\item \texttt{question3\_complete\_restricted\_qOne\_discrete} (discrete time).
\end{enumerate}

Each theorem packages:
nonnegative kernel, stationary distribution, explicit \(F^*/P^*\) formula, and nontrivial transitions.

\section{Lean Theorem Map and Verification Status}
\begin{center}
\small
\begin{tabular}{|p{0.58\linewidth}|p{0.34\linewidth}|}
\hline
\textbf{Lean theorem} & \textbf{Mathematical role} \\
\hline
\texttt{Question3.local\_detailed} &
Local detailed balance identity for adjacent swaps. \\
\hline
\texttt{Question3.stationary\_global\_balance\_explicit} &
Global balance equation from local identities. \\
\hline
\texttt{Question3.transitionRate\_nonneg\_of\_ratePairNonneg} &
Kernel nonnegativity from rate-pair assumptions. \\
\hline
\shortstack[l]{\texttt{Macdonald.Bridge.paper\_main\_restricted\_qOne\_}\\\texttt{discrete\_of\_literature\_assumptions}} &
Conditional literature-interface closure theorem. \\
\hline
\shortstack[l]{\texttt{Macdonald.Bridge.question3\_complete\_}\\\texttt{restricted\_qOne\_discrete}} &
Final discrete stationary-chain theorem. \\
\hline
\shortstack[l]{\texttt{Macdonald.Bridge.question3\_complete\_}\\\texttt{restricted\_qOne}} &
Final continuous-time packaged theorem. \\
\hline
\end{tabular}
\end{center}

Repository verification checklist:
\begin{enumerate}
\item \texttt{lake build} succeeds on the project.
\item no \texttt{sorry}, \texttt{admit}, or custom \texttt{axiom} placeholders appear in project Lean files.
\end{enumerate}

\section{Discussion}
The principal methodological contribution is a clear separation between:
\begin{enumerate}
\item algebraic identities from special-function theory,
\item stochastic consequences for Markov-chain stationarity.
\end{enumerate}
This separation makes the proof transportable:
new polynomial models can be plugged in by proving the interface assumptions, after which the probability theorem follows automatically.

A natural next step is full in-system formalization of interpolation Macdonald polynomial constructions, so that exchange and positivity assumptions become internal derivations rather than external interfaces.

\begin{thebibliography}{99}

\bibitem{Liggett}
T.~M.~Liggett,
\emph{Interacting Particle Systems},
Springer, 1985.

\bibitem{Macdonald}
I.~G.~Macdonald,
\emph{Symmetric Functions and Hall Polynomials},
2nd ed., Oxford University Press, 1995.

\bibitem{Mathlib}
The mathlib Community,
\emph{The Lean Mathematical Library (mathlib4)},
\url{https://github.com/leanprover-community/mathlib4}.

\end{thebibliography}

\end{document}
