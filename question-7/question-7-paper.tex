\documentclass[11pt]{article}

\usepackage[a4paper,margin=1in]{geometry}
\usepackage{amsmath,amsthm,amssymb,mathtools}
\usepackage{hyperref}
\usepackage{enumitem}
\usepackage{booktabs}

\hypersetup{
  colorlinks=true,
  linkcolor=blue,
  citecolor=blue,
  urlcolor=blue
}

\newtheorem{theorem}{Theorem}
\newtheorem{proposition}[theorem]{Proposition}
\newtheorem{corollary}[theorem]{Corollary}
\theoremstyle{definition}
\newtheorem{definition}[theorem]{Definition}
\theoremstyle{remark}
\newtheorem{remark}[theorem]{Remark}

\title{A Deck-Transformation Obstruction to \(2\)-Torsion\\
for Uniform Lattices with \(\mathbb{Q}\)-Acyclic Universal Cover}
\author{Amadeu Zou\\\texttt{amadeuzou@gmail.com}}
\date{February 13, 2026}

\begin{document}
\maketitle

\begin{abstract}
We study the following question (``Question 7'' in the project statement):
can a uniform lattice \(\Gamma\) in a real semisimple group, containing an element of order \(2\), be realized as \(\pi_1(M)\) for a compact boundaryless manifold \(M\) whose universal cover \(\widetilde M\) is \(\mathbb{Q}\)-acyclic?
We isolate the argument into two layers.
The first layer is a purely covering-space obstruction: if every involutive self-homeomorphism of \(\widetilde M\) has a fixed point, then \(\Gamma\) has no \(2\)-torsion.
The second layer is topological: in concrete categories, one supplies the fixed-point input from classical Lefschetz/Smith theory.
This separation makes explicit what is ``heavy topology'' and what is ``formalizable core logic.''
We provide a Lean 4/mathlib formalization of the full first layer and of the interface-level second layer.
The resulting paper is therefore both a mathematical reduction result and a formalization-engineering case study in dependency isolation.
Code and formal artifacts are available at \url{https://github.com/amadeuzou/1stProof-lean4}.
\end{abstract}

\section{Introduction and Motivation}
\subsection*{Origin of the question}
The base problem is recorded in \texttt{question-7.tex} in this repository:
if \(\Gamma\) is a uniform lattice in a real semisimple group and contains \(2\)-torsion, can \(\Gamma\) be the fundamental group of a compact boundaryless manifold with \(\mathbb{Q}\)-acyclic universal cover?

At first sight, experts in transformation groups may view the obstruction as ``morally expected'':
finite-order deck transformations should be constrained by fixed-point principles on acyclic covers.
However, making this precise requires combining ingredients from different domains:
covering-space rigidity, torsion in lattices, and classical fixed-point theory.
The present note makes this reduction explicit and machine-checkable.

\subsection*{Why this is not only a one-line argument}
The core contradiction can indeed be summarized quickly:
``an order-\(2\) deck transformation is involutive, involutions have fixed points, deck transformations with fixed points are trivial.''
But for a research-style treatment, three details matter:
\begin{enumerate}[leftmargin=2em]
\item one must state exactly which fixed-point theorem is used, and in which topological category;
\item one must separate what is classical citation-dependent input from what is formally verified in-repo;
\item one must document the trusted code base (TCB) and dependency boundary in the Lean development.
\end{enumerate}
This paper is written with that standard in mind.

\subsection*{Main contributions}
\begin{enumerate}[leftmargin=2em]
\item A precise obstruction theorem: under an involutive fixed-point property on the universal cover, \(\Gamma\) has no \(2\)-torsion.
\item A topological bridge formulation: Lefschetz/Smith input is isolated as an explicit interface rather than hidden in informal prose.
\item A Lean 4 formalization of the complete obstruction chain outside heavy homological infrastructure, with an auditable verification footprint.
\end{enumerate}

\section{Mathematical Setup and Preliminaries}
Let \(p : E \to M\) be a universal covering map, with \(E\) connected and simply connected.
Assume \(\Gamma\) acts by deck transformations via a faithful homomorphism
\[
\rho : \Gamma \hookrightarrow \mathrm{Deck}(p).
\]
This is the standard realization of \(\Gamma \cong \pi_1(M)\) as deck transformations.

\begin{definition}[\(2\)-torsion]
\(\Gamma\) has \(2\)-torsion if
\[
\exists \gamma \in \Gamma,\quad \operatorname{ord}(\gamma)=2.
\]
\end{definition}

\begin{definition}[Involutive fixed-point property]
For a topological space \(E\), write \((\mathrm{FP})\) for:
\[
\forall g \in \mathrm{Homeo}(E),\quad g^2=\mathrm{id}_E \Longrightarrow \mathrm{Fix}(g)\neq\varnothing.
\]
\end{definition}

\begin{remark}
In this paper, \((\mathrm{FP})\) is the exact topological input required by the obstruction proof.
The Lean development packages this as
\texttt{HasInvolutiveHomeomorphFixedPointProperty E}.
\end{remark}

\section{Topological Bridge: From Acyclicity to Fixed Points}
\subsection*{Classical context (Smith/Lefschetz)}
Classical Smith theory studies finite \(p\)-group actions on mod-\(p\) acyclic spaces and gives strong consequences for fixed sets.
A representative conclusion is:
for a \(\mathbb{Z}/p\)-action on a suitable finite-dimensional mod-\(p\) acyclic space,
the fixed-point set is again mod-\(p\) acyclic, hence nonempty.
For \(p=2\), this gives fixed points for involutions.
See Bredon \cite{Bredon}, especially Chapter III, \S 3 (``Transformations of Prime Period'')
and Chapter III, \S 7 (``Finite Group Actions on General Spaces'').

From another direction, Lefschetz-type theorems provide fixed points via nonvanishing Lefschetz number under finite-type assumptions \cite{Lefschetz,Hatcher}.
In practical applications one chooses the route compatible with the category of \(E\).

\subsection*{Citation map used in this manuscript}
To make the bridge input explicit at reference level, we use the following citation map.
\begin{enumerate}[leftmargin=2em]
\item Lefschetz route: Hatcher, Theorem 2C.3 (Lefschetz Fixed Point Theorem) \cite[Theorem 2C.3]{Hatcher}.
\item Smith route: Bredon, Chapter III, \S 3 and \S 7 for prime-period and finite-group fixed-point consequences; see also Chapter VII, \S 7 (``A Theorem on Involutions'') for involution-focused statements \cite{Bredon}.
\end{enumerate}
Theorem numbering in Bredon is edition/printing sensitive in secondary citations, so we cite by chapter and section.

\subsection*{What is assumed here}
This manuscript does \emph{not} claim that bare ``\(\mathbb{Q}\)-acyclic'' alone implies \((\mathrm{FP})\) in full generality.
Instead, we isolate the needed implication as a bridge hypothesis:
\[
\text{(Bridge)}\qquad
\bigl(\text{chosen classical fixed-point input in the working category}\bigr)
\Longrightarrow (\mathrm{FP}).
\]
In Lean this bridge is represented by the typeclass
\texttt{LefschetzInvolutionFixedPointBridge E},
separating geometric topology dependencies from the deck/group core.

\subsection*{Why this separation is useful}
This design addresses a common formalization bottleneck:
full singular-homology and Smith machinery is substantial, while the obstruction logic is short and reusable.
By isolating the bridge, we can machine-check the entire reduction pipeline now,
and later replace the external bridge by a fully internal proof without changing downstream theorems.

\section{The Obstruction Argument}
\begin{proposition}[Deck fixed-point rigidity]
\label{prop:deck-fixed-identity}
Let \(p:E\to M\) be a covering with \(E\) connected.
If \(d\in \mathrm{Deck}(p)\) has a fixed point, then \(d=\mathrm{id}_E\).
\end{proposition}

\begin{proof}
Deck transformations satisfy \(p\circ d=p\).
Hence both \(d\) and \(\mathrm{id}_E\) are lifts of \(p\).
If they agree at one point (a fixed point of \(d\)), uniqueness of lifts on connected domains implies equality everywhere.
Therefore \(d=\mathrm{id}_E\).
\end{proof}

\begin{theorem}[Obstruction theorem]
\label{thm:obstruction}
Assume \((\mathrm{FP})\) on \(E\).
Then \(\Gamma\) has no \(2\)-torsion.
\end{theorem}

\begin{proof}
Assume \(\Gamma\) has \(2\)-torsion.
Choose \(\gamma\in\Gamma\) with \(\operatorname{ord}(\gamma)=2\), and put \(d:=\rho(\gamma)\in\mathrm{Deck}(p)\).
Since \(\rho\) is a homomorphism,
\[
d^2=\rho(\gamma)^2=\rho(\gamma^2)=\rho(e)=\mathrm{id}_E,
\]
so \(d\) is involutive.
By \((\mathrm{FP})\), \(d\) has a fixed point.
By Proposition~\ref{prop:deck-fixed-identity}, \(d=\mathrm{id}_E\).
Injectivity of \(\rho\) then forces \(\gamma=e\), contradicting \(\operatorname{ord}(\gamma)=2\).
Hence \(\Gamma\) has no \(2\)-torsion.
\end{proof}

\begin{corollary}[Negative answer to Question 7]
\label{cor:q7}
Let \(\Gamma\) be a uniform lattice in a real semisimple group and suppose
\(\Gamma\cong\pi_1(M)\) for a compact boundaryless manifold \(M\) with universal cover \(E\).
If the chosen Lefschetz/Smith bridge for this category yields \((\mathrm{FP})\) on \(E\),
then \(\Gamma\) cannot contain \(2\)-torsion.
\end{corollary}

\begin{proof}
Apply Theorem~\ref{thm:obstruction} to the deck realization.
\end{proof}

\begin{remark}[Interpretation]
The theorem is a reduction principle:
it identifies the exact topological input needed from Smith/Lefschetz theory,
then proves the group-theoretic obstruction independently.
This is the key reason the result is robust under changes of topological category.
\end{remark}

\section{Formal Verification Note}
The Lean development mirrors the mathematical decomposition:
the obstruction proof (deck rigidity + torsion contradiction) is formalized as the core layer,
while the Smith/Lefschetz input is represented as a bridge interface.
Concretely, the fixed-point hypothesis is encoded as
\texttt{HasInvolutiveHomeomorphFixedPointProperty E},
and the acyclicity-to-fixed-point implication is packaged by
\texttt{LefschetzInvolutionFixedPointBridge E}.
This keeps the proof text mathematically standard while making the dependency boundary explicit.

\subsection*{Math-to-Lean correspondence}
Table~\ref{tab:lean-map} records the principal theorem interfaces.

\begin{table}[h]
\centering
\small
\begin{tabular}{@{}p{0.38\linewidth}p{0.56\linewidth}@{}}
\toprule
\textbf{Lean theorem/interface} & \textbf{Mathematical role} \\
\midrule
\texttt{deckTransformation\_eq\_refl\_homeomorph\_of\_fixed\_point} & Proposition~\ref{prop:deck-fixed-identity}. \\
\texttt{no\_two\_torsion\_of\_realization} & Core obstruction from faithful deck action + order-2 fixed-point input. \\
\texttt{question7\_main\_paper\_form\_of\_fixed\_point\_theorem} & Paper-facing theorem with explicit \((\mathrm{FP})\) hypothesis. \\
\texttt{question7\_main\_paper\_form} & Bridge-instance version (Lefschetz/Smith bridge supplied as typeclass). \\
\texttt{question7\_main\_paper\_form\_odd} & Finite odd-cardinality route with automatic bridge instance. \\
\bottomrule
\end{tabular}
\caption{Main interfaces in the Lean development.}
\label{tab:lean-map}
\end{table}

\subsection*{Verification footprint}
All statements below were checked in this repository on February 13, 2026.

\begin{table}[h]
\centering
\small
\begin{tabular}{@{}p{0.52\linewidth}p{0.40\linewidth}@{}}
\toprule
\textbf{Audit item} & \textbf{Result} \\
\midrule
Lean source size (\texttt{Question7.lean} + \texttt{Question7/*.lean}) & 1269 lines \\
Core declarations (\texttt{theorem/lemma/def/class/structure}) & 95 \\
\texttt{lake build} & success \\
Search for \texttt{sorry/admit/axiom} & no matches \\
\texttt{\#print axioms} on final paper entry points & only \texttt{propext}, \texttt{Classical.choice}, \texttt{Quot.sound} \\
\bottomrule
\end{tabular}
\caption{Formalization audit summary.}
\label{tab:audit}
\end{table}

Thus the trusted base is explicit: no project-local axioms and no unfinished placeholders;
the remaining trust is Lean's standard logical kernel assumptions plus the externally cited bridge theorem.

\section{Conclusion and Further Directions}
As a pure mathematical statement, the obstruction theorem is concise.
Its contribution here is not depth-by-length, but precision-by-separation:
we identify the exact topological input needed, prove the rest in a fully machine-checked way, and make the dependency boundary explicit.

For a full ``from-first-principles inside Lean'' treatment, the next step is to internalize the homological fixed-point bridge itself:
formalize the relevant singular-homology infrastructure and instantiate
\texttt{LefschetzInvolutionFixedPointBridge} without external citation.
That would convert this citation-based formalization into a fully internal one.

\begin{thebibliography}{99}

\bibitem{Bredon}
G.~E.~Bredon,
\emph{Introduction to Compact Transformation Groups},
Academic Press, 1972.

\bibitem{Hatcher}
A.~Hatcher,
\emph{Algebraic Topology},
Cambridge University Press, 2002.
\newline See Additional Topic 2.C, Theorem 2C.3.
\newline \url{https://pi.math.cornell.edu/~hatcher/AT/ATpage.html}

\bibitem{Lefschetz}
S.~Lefschetz,
\emph{Algebraic Topology},
American Mathematical Society Colloquium Publications, 1942.

\bibitem{Mathlib}
The mathlib Community,
\emph{The Lean Mathematical Library (mathlib4)},
\url{https://github.com/leanprover-community/mathlib4}.

\bibitem{Lean4}
L.~de Moura et al.,
\emph{The Lean 4 Theorem Prover and Programming Language},
\url{https://lean-lang.org}.

\bibitem{Code}
A.~Zou,
\emph{1stProof-lean4 (Question 7 formalization repository)},
\url{https://github.com/amadeuzou/1stProof-lean4}.

\end{thebibliography}

\end{document}
