\documentclass[11pt]{article}

\usepackage[utf8]{inputenc}
\usepackage[T1]{fontenc}
\usepackage{lmodern}
\usepackage{amsmath,amssymb,amsthm,mathtools}
\usepackage{geometry}
\usepackage{hyperref}
\usepackage{booktabs}
\usepackage{enumitem}
\usepackage{array}
\usepackage{microtype}

\geometry{margin=1in}
\hypersetup{colorlinks=true,linkcolor=blue,citecolor=blue,urlcolor=blue}

\newtheorem{theorem}{Theorem}[section]
\newtheorem{proposition}[theorem]{Proposition}
\newtheorem{lemma}[theorem]{Lemma}
\newtheorem{corollary}[theorem]{Corollary}
\theoremstyle{definition}
\newtheorem{definition}[theorem]{Definition}
\newtheorem{remark}[theorem]{Remark}

\newcommand{\R}{\mathbb{R}}
\newcommand{\C}{\mathbb{C}}
\newcommand{\wstd}{\omega_{\mathrm{std}}}
\newcommand{\lstd}{\lambda_{\mathrm{std}}}

\title{On the Non-Existence of Lagrangian Smoothings for Polyhedral Surfaces in $\R^4$:\\
A Computer-Verified Constructive Counterexample in Lean 4}
\author{Amadeu Zou\\\texttt{amadeuzou@gmail.com}}
\date{February 12, 2026}

\begin{document}
\maketitle

\begin{abstract}
We give a constructive negative answer to a natural smoothing question in symplectic topology:
if a polyhedral Lagrangian surface $K\subset(\R^4,\wstd)$ has exactly four faces meeting at each vertex,
must $K$ admit a Lagrangian smoothing?
We explicitly construct an octahedron-like polyhedral surface in standard symplectic $\R^4$,
verify by exact computation that every triangular face has zero symplectic area, and certify the
required local combinatorics (four faces at each vertex, two faces at each edge, nondegenerate faces).
The non-smoothability argument is organized modularly by isolating the only heavy input---a standard
Gromov-type obstruction (pseudo-holomorphic curves) excluding compact smooth Lagrangian spheres in
$\R^4$---as an explicit hypothesis parameter.
All constructive parts are formally verified in Lean 4.
Code: \url{https://github.com/amadeuzou/1stProof-lean4}.
\end{abstract}

\section{Introduction and Motivation}
Polyhedral Lagrangian objects occur naturally as limits of smooth Lagrangian submanifolds,
for instance under degenerations, surgery constructions, or discretizations.
A fundamental question is whether such non-smooth Lagrangian surfaces can be smoothed
through genuine Lagrangian embeddings.

The problem studied here (from \texttt{question-8.tex}) asks:

\begin{quote}
Let $K$ be a polyhedral Lagrangian surface in $\R^4$ such that exactly $4$ faces meet at every vertex.
Does $K$ necessarily have a Lagrangian smoothing?
\end{quote}

Our main message is that the local combinatorial regularity ``four faces per vertex'' does not
force the existence of a smoothing.
The reason is global: for some $K$ the existence of a smoothing would imply the existence of
a compact smooth Lagrangian $2$-sphere in $(\R^4,\wstd)$, which is ruled out by the classical
pseudo-holomorphic curve theory initiated by Gromov \cite{Gromov1985PseudoHolomorphic}.

A second goal is methodological.
In informal proofs, the check that all faces of a polyhedron are Lagrangian is often written as
``a direct (but tedious) computation.''
We replace this by a \emph{verified computation}: our explicit coordinates are checked symbolically
inside Lean 4 and also cross-checked by an auxiliary script.

\section{Preliminaries}
\subsection{Standard symplectic $\R^4$}
Write $\R^4\cong \R^2\times\R^2$ with coordinates $(x_1,y_1,x_2,y_2)$.
The standard symplectic form is
\[
\wstd = dx_1\wedge dy_1 + dx_2\wedge dy_2.
\]
Equivalently, for vectors $u=(u_0,u_1,u_2,u_3)$ and $v=(v_0,v_1,v_2,v_3)$,
\begin{equation}\label{eq:omega}
\wstd(u,v)=u_0v_1-u_1v_0+u_2v_3-u_3v_2.
\end{equation}
A convenient primitive is a Liouville one-form $\lstd$ with $d\lstd=\wstd$ (any standard choice
suffices for the exactness discussion below).

\subsection{Polyhedral Lagrangian surfaces}
We work with finite $2$-dimensional polyhedral (simplicial) complexes embedded in $\R^4$.

\begin{definition}[Face-Lagrangian condition]
Let $A,B,C\in\R^4$ be vertices of an oriented triangle.
Define its \emph{symplectic area} by
\begin{equation}\label{eq:triangle-area}
\Omega(A,B,C):=\wstd(B-A,\,C-A).
\end{equation}
A planar triangle is called \emph{Lagrangian} if $\Omega(A,B,C)=0$.
A polyhedral surface is \emph{face-Lagrangian} if each of its triangular faces is Lagrangian.
\end{definition}

For an affine triangle in $\R^4$, the Lagrangian condition for its tangent plane is equivalent to
\eqref{eq:triangle-area}; hence this is a correct local check for piecewise linear faces.

\subsection{Lagrangian smoothing}
The problem statement defines a Lagrangian smoothing of $K$ as a Hamiltonian isotopy $K_t$ of
smooth Lagrangian submanifolds for $t\in(0,1]$ extending to a topological isotopy on $[0,1]$
with endpoint $K_0=K$.
In particular, for every $t>0$, $K_t$ is a compact smooth Lagrangian surface homeomorphic to $K$.

\subsection{Gromov-type obstruction (external input)}
The only non-constructive input in the argument is the following classical obstruction.

\begin{theorem}[Gromov, informal consequence]\label{thm:gromov}
There is no compact smooth Lagrangian $2$-sphere in standard symplectic $(\R^4,\wstd)$.
\end{theorem}

\begin{remark}
A standard route to \autoref{thm:gromov} is:
(i) if $L\subset(\R^{2n},\wstd)$ is compact Lagrangian then $\lstd|_L$ is closed since
$d(\lstd|_L)=(d\lstd)|_L=\wstd|_L=0$;
(ii) if $L\cong S^2$ then $H^1(L)=0$, so $\lstd|_L$ is exact and $L$ is an exact Lagrangian;
(iii) Gromov \cite{Gromov1985PseudoHolomorphic} implies there is no compact exact Lagrangian in $\R^{2n}$.
In the Lean formalization we represent \autoref{thm:gromov} by an explicit hypothesis parameter
\(\texttt{NoCompactLagrangianSphereInR4}\) and keep the constructive core independent of it.
\end{remark}

\section{Why the Octahedron?}
A triangulated closed surface in which exactly four triangles meet at every vertex is naturally
modeled by the boundary of an octahedron: it has six vertices and the vertex degree is exactly four.
In this sense, the octahedron is the smallest and most symmetric combinatorial candidate satisfying
``four faces per vertex'' among triangulated spheres.

This motivates using the octahedron combinatorics as the underlying simplicial complex.
The remaining task is to find an embedding into $\R^4$ for which every triangular face is Lagrangian.

\section{Explicit Construction in $\R^4$}
\subsection{Vertices}
We now define six points in $\R^4$:
\begin{equation}\label{eq:verts}
\begin{aligned}
P&=(0,0,0,0), &\qquad N&=(-1,1,-1,1),\\
Q_1&=(1,0,0,0), & Q_2&=(0,0,1,0),\\
Q_3&=(0,1,0,0), & Q_4&=(0,0,0,1).
\end{aligned}
\end{equation}

\subsection{Faces and edges}
We take the eight oriented triangular faces (the suspension of a $4$-cycle):
\begin{equation}\label{eq:faces}
\begin{aligned}
&(P,Q_1,Q_2),(P,Q_2,Q_3),(P,Q_3,Q_4),(P,Q_4,Q_1),\\
&(N,Q_1,Q_2),(N,Q_2,Q_3),(N,Q_3,Q_4),(N,Q_4,Q_1).
\end{aligned}
\end{equation}
The $12$ edges are the obvious pairs connecting $P$ and $N$ to each $Q_i$ and the cycle edges
$(Q_1,Q_2),(Q_2,Q_3),(Q_3,Q_4),(Q_4,Q_1)$.

\subsection{Local incidence properties}
\begin{lemma}[Four faces per vertex]
In the surface defined by \eqref{eq:faces}, exactly four faces are incident to each vertex.
\end{lemma}

\begin{proof}
This is checked directly from the face list: $P$ is incident to the four faces in the first line
of \eqref{eq:faces}, $N$ to the four faces in the second line, and each $Q_i$ appears in exactly four
faces.
In Lean this is certified by an explicit incidence list and completeness proof
(\texttt{octahedron\_four\_faces}).
\end{proof}

\begin{lemma}[Two faces per edge]
Every edge is incident to exactly two faces.
\end{lemma}

\begin{proof}
Each edge of the equatorial cycle $(Q_i,Q_{i+1})$ is contained in exactly two faces, one with apex $P$
and one with apex $N$. Similarly, each edge $(P,Q_i)$ (resp. $(N,Q_i)$) lies in the two faces adjacent to
$Q_i$ in the cycle.
In Lean this is verified by explicit edge-face incidence lists
(\texttt{octahedron\_each\_edge\_has\_two\_incident\_faces}).
\end{proof}

\begin{lemma}[Nondegenerate faces and injective vertex map]
All triangular faces have three distinct vertices and the vertex coordinate map is injective.
\end{lemma}

\begin{proof}
Immediate from the explicit coordinates \eqref{eq:verts}.
In Lean: \texttt{oct\_face\_distinct\_vertices} and \texttt{OctCoords\_injective}.
\end{proof}

\subsection{Topology: the underlying surface is a sphere}
The abstract simplicial complex given by \eqref{eq:faces} is the boundary complex of an octahedron,
thus a triangulation of $S^2$.
For bookkeeping we also record the Euler characteristic:
\[
\chi = V-E+F = 6-12+8 = 2.
\]
In Lean, the combinatorics and local manifold certificates are packaged as
\texttt{OctTopologicalSubmanifoldCertificate}.

\section{Verified Lagrangian Condition for Each Face}
\subsection{Symplectic area computation}
For a face $(A,B,C)$, the Lagrangian condition is $\Omega(A,B,C)=0$ where $\Omega$ is defined
in \eqref{eq:triangle-area}.

\begin{proposition}[All faces are Lagrangian]\label{prop:faces-lag}
For each of the eight faces in \eqref{eq:faces}, the symplectic area vanishes.
Equivalently, the surface is face-Lagrangian.
\end{proposition}

\begin{proof}
This is a direct computation using \eqref{eq:omega} and \eqref{eq:verts}.
For example,
\[
\Omega(P,Q_1,Q_2)=\wstd((1,0,0,0),(0,0,1,0))=0,
\]
and similarly for the other seven faces.

In Lean, we define \texttt{OctFaceArea} and prove
\(\forall f,\ \texttt{OctFaceArea}(f)=0\) by exact symbolic evaluation using
\texttt{fin\_cases} and \texttt{norm\_num}.
The auxiliary script \texttt{scripts/verify\_octa\_lagrangian.py} prints the face list and confirms all
areas are exactly $0$.
\end{proof}

\subsection{Face table}
For reproducibility, we list the faces and their computed symplectic areas.

\begin{table}[h]
\centering
\small
\begin{tabular}{>{\raggedright}p{0.62\linewidth} >{\centering\arraybackslash}p{0.20\linewidth}}
\toprule
Face $(A,B,C)$ & $\Omega(A,B,C)$ \\
\midrule
$(P,Q_1,Q_2)$, $(P,Q_2,Q_3)$, $(P,Q_3,Q_4)$, $(P,Q_4,Q_1)$ & $0$ \\
$(N,Q_1,Q_2)$, $(N,Q_2,Q_3)$, $(N,Q_3,Q_4)$, $(N,Q_4,Q_1)$ & $0$ \\
\bottomrule
\end{tabular}
\caption{All eight faces have vanishing symplectic area.}
\end{table}

\section{Non-Existence of Lagrangian Smoothing}
We now explain why this explicit polyhedral Lagrangian surface cannot be smoothed.

\begin{theorem}[Non-smoothability from the Gromov obstruction]\label{thm:main}
Assume \autoref{thm:gromov}. Then the explicit face-Lagrangian octahedron surface $K$ defined above
admits no Lagrangian smoothing.
Consequently, the universal claim
\[
\forall K,\ (\text{four faces meet at each vertex})\Rightarrow(\text{$K$ has a Lagrangian smoothing})
\]
is false.
\end{theorem}

\begin{proof}
Assume for contradiction that $K$ admits a Lagrangian smoothing $(K_t)_{t\in(0,1]}$.
By definition, for each $t>0$, $K_t$ is a compact smooth Lagrangian surface in $(\R^4,\wstd)$
and is homeomorphic to $K$.
Since the underlying simplicial complex of $K$ is a triangulated sphere,
$K_t$ is a compact smooth Lagrangian $2$-sphere.
This contradicts \autoref{thm:gromov}.
\end{proof}

\section{Lean 4 Formalization and Verification Engineering}
\subsection{Certificate-based formalization}
Rather than developing a general-purpose PL-manifold library, we use an explicit
\emph{checkable certificate} for the octahedral surface.
Concretely, we prove inside Lean:
\begin{itemize}
\item four faces per vertex via explicit incidence lists;
\item two faces per edge via explicit edge incidence;
\item nondegenerate faces and injective vertex map;
\item Euler characteristic $\chi=2$ from explicit counts;
\item vanishing symplectic areas for all faces.
\end{itemize}
These are assembled into \texttt{OctTopologicalSubmanifoldCertificate}.

\subsection{Automation: verified computations}
The Lagrangian checks are performed with Lean's numeric normalization
(\texttt{norm\_num}) over explicit coordinates.
This replaces informal ``tedious computations'' by proof terms that compile.

\subsection{Architecture: parameterizing the heavy theorem}
The core file \texttt{Question8/Core.lean} contains no custom axioms.
The Gromov input is represented as a hypothesis
\(h\_{\mathrm{NoSphere}}:\texttt{NoCompactLagrangianSphereInR4}\), and the main contradiction
lemmas are proved conditionally.
The file \texttt{Question8/ExternalGromov.lean} provides citation-facing theorem names
(e.g. \texttt{Question8\_Answer}) that explicitly take this hypothesis parameter.
This makes the boundary between constructive computation and external symplectic topology precise.

\subsection{AI-assisted workflow}
The project was completed in an AI-assisted workflow used for:
(i) decomposing the goal into small lemmas,
(ii) proposing explicit coordinate candidates,
(iii) iteratively repairing Lean proof scripts until \texttt{lake build} succeeded,
and (iv) producing reproducibility scripts and manuscript text.

\section{Discussion and Open Problems}
\subsection{Beyond the sphere case}
The obstruction used here is specific to the spherical topology.
For other topological types, e.g. Lagrangian tori in $\R^4$, Gromov's theorem does not rule out
existence, and smoothing questions may have a different answer.
A natural open direction is:
\begin{quote}
If $K$ is a polyhedral face-Lagrangian torus in $\R^4$ with four faces meeting at each vertex,
when does it admit a Lagrangian smoothing?
\end{quote}

\subsection{Oracle-based formalization}
This development illustrates a general formalization strategy:
keep constructive geometry fully verified, and isolate deep external theorems as explicit parameters
(or cited theorems) with minimal interfaces.
This resembles formal workflows in other problems where a heavy external result (e.g. Smith theory)
serves as an ``oracle'' in the final contradiction.

\section*{Reproducibility}
The repository \url{https://github.com/amadeuzou/1stProof-lean4} contains the Lean files and scripts.
A one-shot verification command is:
\begin{verbatim}
./scripts/verify_formal_status.sh
\end{verbatim}
It runs \texttt{lake build}, checks absence of \texttt{axiom}/\texttt{sorry}/\texttt{admit},
and executes a geometry sanity checker.

\bibliographystyle{plain}
\bibliography{question-8-references}

\end{document}
